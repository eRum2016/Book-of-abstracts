\providecommand{\main}{..} 
\documentclass[\main/boa.tex]{subfiles}

\begin{document}
\pagestyle{empty}

\section{How to use R to hack the publicly available data about skills of 2M+
worldwide students?}

\begin{center}
  {\bf Przemyslaw Biecek$^{1^\star}$}
\end{center}

\vskip 0.3cm

\begin{affiliations}
\begin{enumerate}
\begin{minipage}{0.915\textwidth}
\centering
\item University of Warsaw \\[-2pt]
\end{minipage}
\end{enumerate}
$^\star$Contact author: \href{mailto:przemyslaw.biecek@gmail.com}{\nolinkurl{przemyslaw.biecek@gmail.com}}\\
\end{affiliations}

\vskip 0.5cm

\begin{minipage}{0.915\textwidth}
\keywords visualisation; survey data; archivist; data mining
\packages ggplot2; shiny; intsvy; archivist; knitr; BetaBit; PISA2012lite
\end{minipage}

\vskip 0.8cm

During the talk I will introduce The Programme for International Student
Assessment (PISA), an international survey that aims to evaluate
education systems worldwide. It's a source of large data about
educational performance and various other characteristics of over 2 000
000 students from 62 countries. The data from the last PISA assessment
is available in the \textbf{R} package.

To play with it we will use packages and to explain the relation between
parental occupation and student's performance. Then we will overview the
package -- the toolbox for statistical analyses of international
surveys. Finally we will discuss applications of packages and and the
role of reproducibility and traceability of results. At the end I will
introduce the project that aims to boost data science skills of
students.

\end{document}
