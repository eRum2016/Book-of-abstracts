\providecommand{\main}{..} 
\documentclass[\main/boa.tex]{subfiles}

\begin{document}
\pagestyle{empty}

\section{Genie: A new, fast, and outlier-resistant hierarchical clustering
algorithm and its R interface}

\begin{center}
  {\bf Marek Gągolewski$^{1^\star}$}
\end{center}

\vskip 0.3cm

\begin{affiliations}
\begin{enumerate}
\begin{minipage}{0.915\textwidth}
\centering
\item Systems Research Institute, Polish Academy of Sciences \\[-2pt]
\end{minipage}
\end{enumerate}
$^\star$Contact author: \href{mailto:marek@gagolewski.com}{\nolinkurl{marek@gagolewski.com}}\\
\end{affiliations}

\vskip 0.5cm

\begin{minipage}{0.915\textwidth}
\keywords hierarchical clustering; single linkage; inequity measures; Gini-index
\packages genie
\end{minipage}

\vskip 0.8cm

The time needed to apply a hierarchical clustering algorithm is most
often dominated by the number of computations of a pairwise
dissimilarity measure. Such a constraint, for larger data sets, puts at
a disadvantage the use of all the classical linkage criteria but the
single linkage one. However, it is known that the single linkage
clustering algorithm is very sensitive to outliers, produces highly
skewed dendrograms, and therefore usually does not reflect the true
underlying data structure - unless the clusters are well-separated.

To overcome its limitations, we proposed a new hierarchical clustering
linkage criterion called \emph{genie}. Namely, our algorithm links two
clusters in such a way that a chosen economic inequity measure (e.g.,
the Gini or Bonferroni index) of the cluster sizes does not increase
drastically above a given threshold.

Benchmarks indicate a high practical usefulness of the introduced
method: it most often outperforms the Ward or average linkage in terms
of the clustering quality while retaining the single linkage speed. The
algorithm is easily parallelizable and thus may be run on multiple
threads to speed up its execution further on. Its memory overhead is
small: there is no need to precompute the complete distance matrix to
perform the computations in order to obtain a desired clustering. In
this talk we will discuss its reference implementation, included in the
\emph{genie} package for \textbf{R}.

\end{document}
