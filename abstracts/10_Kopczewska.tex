\documentclass[11pt, a4paper]{article}
\usepackage[utf8]{inputenc}
\usepackage[T1]{fontenc}
\usepackage{eurosym}
\usepackage{amsfonts, amsmath, hanging, hyperref, parskip, times}
\usepackage[numbers]{natbib}
\usepackage[pdftex]{graphicx}
\hypersetup{
  colorlinks,
  linkcolor=black,
  urlcolor=black,
  citecolor=black
}

\let\section=\subsubsection
\newcommand{\pkg}[1]{{\normalfont\fontseries{b}\selectfont #1}}
\let\proglang=\textit
\let\code=\texttt
\renewcommand{\title}[1]{\begin{center}{\bf \LARGE #1}\end{center}}
\newcommand{\affiliations}{\footnotesize\centering}
\newcommand{\keywords}{\paragraph{Keywords:}}
\newcommand{\packages}{\paragraph{R packages:}}

\providecommand{\tightlist}{%
  \setlength{\itemsep}{0pt}\setlength{\parskip}{0pt}}

\setlength{\topmargin}{-15mm}
\setlength{\oddsidemargin}{-2mm}
\setlength{\textwidth}{165mm}
\setlength{\textheight}{250mm}


\begin{document}
\pagestyle{empty}

\title{Geo-located point data: measurement of agglomeration and concentration}

\begin{center}
  {\bf Katarzyna Kopczewska$^{1^\star}$}
\end{center}

\vskip 0.3cm

\begin{affiliations}
\begin{enumerate}
\begin{minipage}{0.915\textwidth}
\centering
\item Faculty of Economic Sciences, Unievrsity of Warsaw \\[-2pt]
\end{minipage}
\end{enumerate}
$^\star$Contact author: \href{mailto:kkopczewska@wne.uw.edu.pl}{\nolinkurl{kkopczewska@wne.uw.edu.pl}}\\
\end{affiliations}

\vskip 0.5cm

\begin{minipage}{0.915\textwidth}
\keywords spatial location; agglomeration; specialization; SPAG; Ripley's K
\packages rgeos; spdep; sp; rgdal; dbmss
\end{minipage}

\vskip 0.8cm

Geo-located points, representing locations of business and other units
can be analysed with regard to agglomeration and concentration patterns,
which in fact indicate a density of region's coverage with the economic
activity. There is plenty of measures on territorially aggregated data
(so called cluster-based measures) and just few on geo-located
individual data (so-called distance-based measures) as SPAG or Ripley's
\(K\). This is to present current possibilities of statistical analysis
of geo-located data in \textbf{R}, as well its applications in regional
science. It would consider the sensitivity of measures for different
spatial patterns. Spatial package \emph{rgeos} allows for treating
points as geometries, what expands significantly the analytical
capabilities. One can also see the neighbourhood relations between
points, which can be applied further in econometric models.

\end{document}
