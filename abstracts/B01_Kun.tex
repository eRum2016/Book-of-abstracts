\providecommand{\main}{..} 
\documentclass[\main/boa.tex]{subfiles}

\begin{document}

\section{Enterprise R Platform -- the what, the why and the how}

\begin{center}
  {\bf \index[a]{Gergely Mark}$^{1}$, \index[a]{Csongor Somogyi}$^{1}$, \index[a]{David Kun}$^{1^\star}$}
\end{center}

\vskip 0.3cm

\begin{affiliations}
\begin{enumerate}
\begin{minipage}{0.915\textwidth}
\centering
\item ownr.io; Functional Finances Ltd \\[-2pt]
\end{minipage}
\end{enumerate}
$^\star$Contact author: \href{mailto:david.kun@functionalfinances.com}{\nolinkurl{david.kun@functionalfinances.com}}\\
\end{affiliations}

\vskip 0.5cm

\begin{minipage}{0.915\textwidth}
\keywords deployment; process; control; enterprise
\packages base; roveR; shiny
\end{minipage}

\vskip 0.8cm

In this talk we will introduce a concept for an Enterprise \textbf{R}
Platform, starting with expectations towards such a platform, followed
by the benefits and finally giving a reference implementation
architecture and a case study. \textbf{R} users in enterprise settings
are often building shadow IT, with no controls but also no access to
efficient code sharing, automation, deployment, version control, etc.
This also means that others in the enterprise cannot benefit from the
\textbf{R}-based solutions. Our proposed architecture addresses all of
these issues by introducing a playground for the \textbf{R} users and a
deployment process supporting all the flexibility needed while subject
to regular controls. Our solution also includes a way to separate the
\textbf{R} projects within a single environment, so different versions
of the same package can be installed for instance. Finally, by providing
a REST API, we enable non-\textbf{R} users to embed the \textbf{R}-based
tools in any other application in the enterprise, ranging from MS Excel
via Java-based ETL tools like Informatica and MI/BI tools like Business
Objects up to and including in-house solutions created in other
languages.

\end{document}
