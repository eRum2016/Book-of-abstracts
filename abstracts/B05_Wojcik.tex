\providecommand{\main}{..} 
\documentclass[\main/boa.tex]{subfiles}

\begin{document}
\pagestyle{empty}

\section{Using R for backtesting algorithmic trading strategies on high-frequency
data}

\begin{center}
  {\bf Piotr Wójcik$^{1^\star}$}
\end{center}

\vskip 0.3cm

\begin{affiliations}
\begin{enumerate}
\begin{minipage}{0.915\textwidth}
\centering
\item Faculty of Economic Sciences, University of Warsaw \\[-2pt]
\end{minipage}
\end{enumerate}
$^\star$Contact author: \href{mailto:pwojcik@wne.uw.edu.pl}{\nolinkurl{pwojcik@wne.uw.edu.pl}}\\
\end{affiliations}

\vskip 0.5cm

\begin{minipage}{0.915\textwidth}
\keywords algorithmic trading; high-frequency data; backtesting; efficient
calculations
\packages xts; chron; quantmod; Rbbg; IBrokers; TFX; tseries;
PerformanceAnalytics; caTools; TTR; inline; Rcpp; RcppArmadillo
\end{minipage}

\vskip 0.8cm

Along with the advances in computer technology high-frequency trading
developed in 1990s and became widely popular since then. Traders started
to build algorithms that use highly developed quantitative models to
automatically determine when and where to trade. The profitability of
trading strategies based on such algorithms needs to be verified on
historical data (backtested) prior to its application in real life.
Intraday data in finance may have large volumes (up to 1440 minutes or
86400 seconds of quotations every day, i.e.~20 millions of observations
for one year). In addition, developed algorithms are usually
parametrized and their final version needs to be optimized with respect
to these parameters. This introduces a possibly huge number of
combinations that need to be compared with respect to selected
performance measures. Large amounts of data and many variants of the
algorithm require a computationally efficient tool that should also
allow to relatively easily apply statistical models. And \textbf{R}
together with C++ can provide such a tool. The presentation shows how
\textbf{R} can be used to access intraday data and to develop and
backtest different algorithmic trading strategies with the help of
\emph{Rcpp} family packages.

\end{document}
