\providecommand{\main}{..} 
\documentclass[\main/boa.tex]{subfiles}

\begin{document}

\section{Reconciling forecasts: the hts package}

\begin{center}
  {\bf \index[a]{Hyndman Rob} Rob Hyndman$^{1^\star}$}
\end{center}

\vskip 0.3cm

\begin{affiliations}
\begin{enumerate}
\begin{minipage}{0.915\textwidth}
\centering
\item Monash University \\[-2pt]
\end{minipage}
\end{enumerate}
$^\star$Contact author: \href{mailto:Rob.Hyndman@monash.edu}{\nolinkurl{Rob.Hyndman@monash.edu}}\\
\end{affiliations}

\vskip 0.5cm

\begin{minipage}{0.915\textwidth}
\keywords forecast; time series
\packages \index[p]{hts} hts
\end{minipage}

\vskip 0.8cm

Hierarchical time series occur when there are multiple time series that
are hierarchically organized and can be aggregated at several different
levels based on dimensions such as product, geography, or some other
features. A common application occurs in manufacturing where forecasts
of sales need to be made for a range of different products in different
locations. The forecasts need to add up appropriately across the levels
of the hierarchy. I will describe some new features in the \emph{hts}
package for \textbf{R} which provides several methods for analysing and
forecasting hierarchical and grouped time series.

\end{document}
