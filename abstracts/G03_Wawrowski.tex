\providecommand{\main}{..} 
\documentclass[\main/boa.tex]{subfiles}

\begin{document}
\pagestyle{empty}

\section{R packages for social indicators}

\begin{center}
  {\bf Łukasz Wawrowski$^{1^\star}$}
\end{center}

\vskip 0.3cm

\begin{affiliations}
\begin{enumerate}
\begin{minipage}{0.915\textwidth}
\centering
\item Poznań University of Economics and Business \\[-2pt]
\end{minipage}
\end{enumerate}
$^\star$Contact author: \href{mailto:lukasz8989@gmail.com}{\nolinkurl{lukasz8989@gmail.com}}\\
\end{affiliations}

\vskip 0.5cm

\begin{minipage}{0.915\textwidth}
\keywords sample surveys; social indicators
\packages survey; laeken; ineq; vardpoor; convey
\end{minipage}

\vskip 0.8cm

Social cohesion is a very popular motto in European Union. It is
measured by many different indicators such as poverty rate, low work
intensity indicator or quintile share ratio. They were established by
the European Union as part of the Lisbon Strategy and the Europa 2020
Strategy. Its function is to control realization of above mentioned
strategies. Presentation aims to compare \textbf{R} packages used in
social indicators estimation. Features and efficiency of different
packages will be examined.

\end{document}
