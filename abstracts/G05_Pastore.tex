\providecommand{\main}{..} 
\documentclass[\main/boa.tex]{subfiles}

\begin{document}
\pagestyle{empty}

\section{influence.SEM 2.0: An R Package for Sensitivity Analysis in Structural
Equation Models}

\begin{center}
  {\bf Gianmarco Altoè$^{1}$, Massimo Nucci$^{1}$, Massimiliano Pastore$^{1^\star}$}
\end{center}

\vskip 0.3cm

\begin{affiliations}
\begin{enumerate}
\begin{minipage}{0.915\textwidth}
\centering
\item University of Padova \\[-2pt]
\end{minipage}
\end{enumerate}
$^\star$Contact author: \href{mailto:massimiliano.pastore@unipd.it}{\nolinkurl{massimiliano.pastore@unipd.it}}\\
\end{affiliations}

\vskip 0.5cm

\begin{minipage}{0.915\textwidth}
\keywords sensitivity analysis; influential cases; structural equation models
\packages lavaan; influence.SEM
\end{minipage}

\vskip 0.8cm

The \textbf{R} package \emph{influence.SEM} 2.0 provides tools to
perform sensitivity analysis in Structural Equation Models (SEM).
Despite SEM are widely used by researchers in the social and behavioral
sciences, the application of associated case-diagnostic tools
(i.e.~sensitivity analysis) have received limited attention and
understanding in practice. Sensitivity analysis provides information
regarding the impact of single cases on parameter estimates and goodness
of fit of the model, thus supporting the researcher in the
interpretation of results. In this paper, we present an easy-to-use
\textbf{R} package yielding several measures of case influence for SEM
via two applications to real data. We discuss the utility of detecting
influential cases in SEM, provide recommendations for the use of
measures of case influence, and give suggestions on how this analysis
can be usefully employed in combination with other statistical
techniques.

\end{document}
