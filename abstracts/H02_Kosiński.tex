\documentclass[11pt, a4paper]{article}
\usepackage[utf8]{inputenc}
\usepackage[T1]{fontenc}
\usepackage{eurosym}
\usepackage{amsfonts, amsmath, hanging, hyperref, parskip, times}
\usepackage[numbers]{natbib}
\usepackage[pdftex]{graphicx}
\hypersetup{
  colorlinks,
  linkcolor=black,
  urlcolor=black,
  citecolor=black
}

\let\section=\subsubsection
\newcommand{\pkg}[1]{{\normalfont\fontseries{b}\selectfont #1}}
\let\proglang=\textit
\let\code=\texttt
\renewcommand{\title}[1]{\begin{center}{\bf \LARGE #1}\end{center}}
\newcommand{\affiliations}{\footnotesize\centering}
\newcommand{\keywords}{\paragraph{Keywords:}}
\newcommand{\packages}{\paragraph{R packages:}}

\providecommand{\tightlist}{%
  \setlength{\itemsep}{0pt}\setlength{\parskip}{0pt}}

\setlength{\topmargin}{-15mm}
\setlength{\oddsidemargin}{-2mm}
\setlength{\textwidth}{165mm}
\setlength{\textheight}{250mm}


\begin{document}
\pagestyle{empty}

\title{archivist 2.0: News from Managing Data Analysis Results Toolkit}

\begin{center}
  {\bf Marcin Kosiński$^{1^\star}$}
\end{center}

\vskip 0.3cm

\begin{affiliations}
\begin{enumerate}
\begin{minipage}{0.915\textwidth}
\centering
\item Warsaw University of Technology \\[-2pt]
\end{minipage}
\end{enumerate}
$^\star$Contact author: \href{mailto:marcin.kosinski@grupawp.pl}{\nolinkurl{marcin.kosinski@grupawp.pl}}\\
\end{affiliations}

\vskip 0.5cm

\begin{minipage}{0.915\textwidth}
\keywords reproducible research; sharing objects; archiving
\packages archivist; archivist.github
\end{minipage}

\vskip 0.8cm

Open science needs not only reproducible research but also accessible
final and partial results. During the speech I will present the most
valuable applications of the \emph{archivist} package. The archivist is
an \textbf{R} package for data analysis results management, which helps
in managing, sharing, storing, linking and searching for \textbf{R}
objects. The \emph{archivist} package automatically retrieves the
object's meta-data and creates a rich structure that allows for easy
management of calculated \textbf{R} objects. The \emph{archivist}
package extends the reproducible research paradigm by creating new ways
to retrieve and validate previously calculated objects. These
functionalities also result in a variety of opportunities such as:
sharing \textbf{R} objects within reports/articles by adding hooks to
\textbf{R} objects in table/figure captions; interactive exploration of
object repositories; caching function calls; retrieving object's
pedigree along with information on how the object was created; automated
tracking of performance of models.

\end{document}
