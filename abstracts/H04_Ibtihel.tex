\providecommand{\main}{..} 
\documentclass[\main/boa.tex]{subfiles}

\begin{document}

\section{An R package for consumer and sensory data mapping: Comparison of Preference Mapping stability}

\begin{center}
  {\bf \index[a]{Ibtihel Rebhi} Rebhi Ibtihel$^{1^\star}$}
\end{center}

\vskip 0.3cm

\begin{affiliations}
\begin{enumerate}
\begin{minipage}{0.915\textwidth}
\centering
\item National Engineering School of Tunis \\[-2pt]
\end{minipage}
\end{enumerate}
$^\star$Contact author: \href{mailto:ibtihelrebhi@yahoo.fr}{\nolinkurl{ibtihelrebhi@yahoo.fr}}\\
\end{affiliations}

\vskip 0.5cm

\begin{minipage}{0.915\textwidth}
\keywords sensory analysis; external preference mapping; comparison of maps stability; assessment of predictions performance; functions for denoising; shiny application
functions
\end{minipage}

\vskip 0.8cm

We propose a new package for analyzing consumer and sensory data dedicated to both academic and industrial sensory analysts. This package implemented in R programming environement, is an easy solution that deals with the following problems: panel performance
evaluation, characterizing products from both sensory and consumer data, mapping of instrumental (sensory or physico-chemical) and consumer data via mapping technics (Internal and External PrefMap), and much emphasis on comparisons of maps stability especially when denoising hedonic data. This variation on PrefMap is also implemented through assessing the performance of different predictions models such as GAM, LM ... and various multivariate analysis methods.

External Preference Mapping functions and Comparison of Maps Stability functions are provided. The shiny application for the functionalities forms part of the package.

This package produces also graphical displays of data that are simple to interpret and it also provides syntheses of results issuing from various ANOVA analysis or various factor analysis methods.

\end{document}
