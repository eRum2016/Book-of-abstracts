\providecommand{\main}{..} 
\documentclass[\main/boa.tex]{subfiles}

\begin{document}

\section{R as a tool for graphical diagnostics in population pharmacokinetic
modeling}

\begin{center}
  {\bf \index[a]{Borsuk Agnieszka} Agnieszka Borsuk$^{1^\star}$}
\end{center}

\vskip 0.3cm

\begin{affiliations}
\begin{enumerate}
\begin{minipage}{0.915\textwidth}
\centering
\item Department of Biopharmaceutics and Pharmacodynamics, Medical University
of Gdańsk \\[-2pt]
\end{minipage}
\end{enumerate}
$^\star$Contact author: \href{mailto:borsuk.agnieszka@gmail.com}{\nolinkurl{borsuk.agnieszka@gmail.com}}\\
\end{affiliations}

\vskip 0.5cm

\begin{minipage}{0.915\textwidth}
\keywords graphical model diagnostics; population pharmacokinetics
\packages \index[p]{ggplot2} ggplot2; \index[p]{lattice}
\end{minipage}

\vskip 0.8cm

Population pharmacokinetic (PopPK) modeling aims at finding typical
pharmacokinetic parameters and their variability within a target
population of patients treated with a drug of interest. Population
approach to pharmacokinetic modeling is gaining popularity as it can
handle sparse data and estimate pharmacokinetic parameters of each
individual. Large datasets and complexity of the models hinder assessing
the quality of a model fit with a single numerical value. Graphical
analysis plays a unique role in PopPK since it enables a better insight
into the model structure. \textbf{R} is a powerful and versatile tool
for graphical model diagnostics. It offers variety of visualizations,
allows creating flexible scripts for quick model assessment and saving
the results as a formal report. The produced graphics are
publication-ready and meet the requirements for most non-standard and
customized plots.

\end{document}
