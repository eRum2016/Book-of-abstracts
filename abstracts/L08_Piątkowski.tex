\documentclass[11pt, a4paper]{article}
\usepackage[utf8]{inputenc}
\usepackage[T1]{fontenc}
\usepackage{eurosym}
\usepackage{amsfonts, amsmath, hanging, hyperref, parskip, times}
\usepackage[numbers]{natbib}
\usepackage[pdftex]{graphicx}
\hypersetup{
  colorlinks,
  linkcolor=black,
  urlcolor=black,
  citecolor=black
}

\let\section=\subsubsection
\newcommand{\pkg}[1]{{\normalfont\fontseries{b}\selectfont #1}}
\let\proglang=\textit
\let\code=\texttt
\renewcommand{\title}[1]{\begin{center}{\bf \LARGE #1}\end{center}}
\newcommand{\affiliations}{\footnotesize\centering}
\newcommand{\keywords}{\paragraph{Keywords:}}
\newcommand{\packages}{\paragraph{R packages:}}

\providecommand{\tightlist}{%
  \setlength{\itemsep}{0pt}\setlength{\parskip}{0pt}}

\setlength{\topmargin}{-15mm}
\setlength{\oddsidemargin}{-2mm}
\setlength{\textwidth}{165mm}
\setlength{\textheight}{250mm}


\begin{document}
\pagestyle{empty}

\title{Structural bioinformatician's notebooks with pdbeeR and knitr}

\begin{center}
  {\bf Paweł Piątkowski$^{1^\star}$}
\end{center}

\vskip 0.3cm

\begin{affiliations}
\begin{enumerate}
\begin{minipage}{0.915\textwidth}
\centering
\item International Institute of Molecular and Cell Biology \\[-2pt]
\end{minipage}
\end{enumerate}
$^\star$Contact author: \href{mailto:ppiatkowski@genesilico.pl}{\nolinkurl{ppiatkowski@genesilico.pl}}\\
\end{affiliations}

\vskip 0.5cm

\begin{minipage}{0.915\textwidth}
\keywords bioinformatics; biomolecules; knitr; reproducible research
\packages pdbeeR; knitr
\end{minipage}

\vskip 0.8cm

Working with biological molecules often involves much work with .pdb
files -- analyzing, subsetting, transforming and visualizing structural
data. Doing this by hand is tedious, hard to reproduce and prone to
errors. A new \textbf{R} package -- \emph{pdbeeR} -- can help you make
these chores fun and save your work as elegant knitr notebooks.

\end{document}
