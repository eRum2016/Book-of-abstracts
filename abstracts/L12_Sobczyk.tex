\providecommand{\main}{..} 
\documentclass[\main/boa.tex]{subfiles}

\begin{document}
\pagestyle{empty}

\section{Visualizing changes in demographics with R}

\begin{center}
  {\bf Piotr Sobczyk$^{1^\star}$}
\end{center}

\vskip 0.3cm

\begin{affiliations}
\begin{enumerate}
\begin{minipage}{0.915\textwidth}
\centering
\item Wrocław University of Technology \\[-2pt]
\end{minipage}
\end{enumerate}
$^\star$Contact author: \href{mailto:pj.sobczyk@gmail.com}{\nolinkurl{pj.sobczyk@gmail.com}}\\
\end{affiliations}

\vskip 0.5cm

\begin{minipage}{0.915\textwidth}
\keywords visualization, demographics
\packages plotly; ggplot2; leaflet
\end{minipage}

\vskip 0.8cm

How to show aging of society? What is a good way to visualize population
projections? How to reveal inadequate public investments related to the
process of suburbanization? Data visualization is a perfect tool to
bring intuitions about those complex and massive problems closer to the
people. In my lighting talk I explore several visualization challenges
related to demographics. For each I propose a solution from my blog
(\url{http://www.szychtawdanych.pl}). Examples I show are created with
\textbf{R} packages \emph{ggplot2}, \emph{plotly} and \emph{leaflet}.
Codes are publicly available on my github
\url{http://www.github.com/psobczyk}.

\end{document}
