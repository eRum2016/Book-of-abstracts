\providecommand{\main}{..} 
\documentclass[\main/boa.tex]{subfiles}

\begin{document}

\section{Mortality mapping using R}

\begin{center}
  {\bf \index[a]{Rabczenko Daniel} Daniel Rabczenko$^{1^\star}$, \index[a]{Rubikowska Barbara} Barbara Rubikowska$^{1}$, \index[a]{Wojtyniak Bogdan} Bogdan Wojtyniak$^{1}$}
\end{center}

\vskip 0.3cm

\begin{affiliations}
\begin{enumerate}
\begin{minipage}{0.915\textwidth}
\centering
\item National Institute of Public Health - National Institute of Hygiene \\[-2pt]
\end{minipage}
\end{enumerate}
$^\star$Contact author: \href{mailto:daniel@pzh.gov.pl}{\nolinkurl{daniel\_@pzh.gov.pl}}\\
\end{affiliations}

\vskip 0.5cm

\begin{minipage}{0.915\textwidth}
\keywords spatial epidemiology, disease cluster
\packages \index[p]{R2WinBUGS} R2WinBUGS; \index[p]{INLA} INLA; \index[p]{spdep} spdep; \index[p]{GWR} GWR
\end{minipage}

\vskip 0.8cm

Analyses of spatial distribution of mortality are important tool for searching of risk factors and as a starting point for preventing actions. Several techniques used for more deep analysis of spatial disease pattern are presented. Due to small number of in poviats deaths mortality characteristics shows large variability. Bayesian Besag-York-Mollie model was used for general spatial trend modelling and producing smooth disease maps. Local Moran coefficient was used to detect clusters of increased mortality. Geographically weighted regression technique as used for inspecting dependence between mortality and deprivation index in different county regions. This work was financed by research grant PL13 Reducing Social Inequalities in Health.

\end{document}
